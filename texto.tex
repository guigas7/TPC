% !TEX encoding = UTF-8 Unicode
%%%%%%%%%%%%%%%%%%%%%%%%%%%%%%%%%%%%%%%%%
% Journal Article
% LaTeX Template
% Version 1.4 (15/5/16)
%
% This template has been downloaded from:
% http://www.LaTeXTemplates.com
%
% Original author:
% Frits Wenneker (http://www.howtotex.com) with extensive modifications by
% Vel (vel@LaTeXTemplates.com)
%
% License:
% CC BY-NC-SA 3.0 (http://creativecommons.org/licenses/by-nc-sa/3.0/)
%
%%%%%%%%%%%%%%%%%%%%%%%%%%%%%%%%%%%%%%%%%

%----------------------------------------------------------------------------------------
%	PACKAGES AND OTHER DOCUMENT CONFIGURATIONS
%----------------------------------------------------------------------------------------
\documentclass[oneside, 11 pt]{article}
\usepackage[utf8]{inputenc}
\usepackage[T1]{fontenc}
\usepackage{textcomp}
\usepackage[portuguese]{babel}
\usepackage{blindtext} % Package to generate dummy text throughout this template 
\usepackage{comment}
\usepackage{listings}
\usepackage{xcolor}
\usepackage{hyperref} % For hyperlinks in the PDF

\usepackage{inconsolata}
\lstset{
	language=bash, %% Troque para PHP, C, Java, etc... bash é o padrão
	basicstyle=\ttfamily\small,
	numberstyle=\footnotesize,
	backgroundcolor=\color{gray!10},
	frame=single,
	tabsize=2,
	rulecolor=\color{black!30},
	escapeinside={\%*}{*)},
	breaklines=true,
	breakatwhitespace=true,
	framextopmargin=2pt,
	framexbottommargin=2pt,
	inputencoding=utf8,
	extendedchars=true,
	literate={á}{{\'a}}1 {ã}{{\~a}}1 {é}{{\'e}}1 {í}{{\'i}}1,
}
\usepackage[hmarginratio=1:1,top=32mm,columnsep=20pt]{geometry} % Document margins
\usepackage[hang, small,labelfont=bf,up,textfont=it,up]{caption} % Custom captions under/above floats in tables or figures
\usepackage{booktabs} % Horizontal rules in tables

\usepackage{lettrine} % The lettrine is the first enlarged letter at the beginning of the text

\usepackage{enumitem} % Customized lists
\setlist[itemize]{noitemsep} % Make itemize lists more compact

\usepackage{titlesec} % Allows customization of titles
%\renewcommand\thesection{\Roman{section}} % Roman numerals for the sections
%\renewcommand\thesubsection{\roman{subsection}} % roman numerals for subsections
\titleformat{\section}[block]{\large\scshape\centering}{\thesection.}{1em}{} % Change the look of the section titles
\titleformat{\subsection}[block]{\large}{\thesubsection.}{1em}{} % Change the look of the section titles

\usepackage{fancyhdr} % Headers and footers
\pagestyle{fancy} % All pages have headers and footers
\fancyhead{} % Blank out the default header
\fancyhead[C]{Como não perder tempo com o mouse em interfaces gráficas} % Custom header text

\usepackage{titling} % Customizing the title section

\usepackage{graphicx}
\usepackage{float}
\renewcommand{\arraystretch}{1.5}
\usepackage{multirow}
\setlength{\parindent}{18pt}
\setcounter{secnumdepth}{0}
\usepackage{tabularx}
	\newcolumntype{L}{>{\raggedright\arraybackslash}X}

\usepackage{xcolor}
\hypersetup{
	colorlinks,
	linkcolor={red!60!black},
	citecolor={blue!50!black},
	urlcolor={blue!80!black}
}
%----------------------------------------------------------------------------------------
%	TITLE SECTION
%----------------------------------------------------------------------------------------

\setlength{\droptitle}{-4\baselineskip} % Move the title up

\pretitle{\begin{center}\Huge\bfseries} % Article title formatting
	\posttitle{\end{center}} % Article title closing formatting
\title{Como não perder tempo com o mouse em interfaces gráficas} % Article title
\author{%
	\textsc{Guilherme Bittencourt Bueno da Silva} \\[1ex] % Your name
	\normalsize Universidade Federal do Paraná \\ % Your institution
	\normalsize {gbbs14@inf.ufpr.br} % Your email address
	%\and % Uncomment if 2 authors are required, duplicate these 4 lines if more
	%\textsc{Jane Smith}\thanks{Corresponding author} \\[1ex] % Second author's name
	%\normalsize University of Utah \\ % Second author's institution
	%\normalsize \href{mailto:jane@smith.com}{jane@smith.com} % Second author's email address
}
\date{\today} % Leave empty to omit a date
\renewcommand{\maketitlehookd}{%
	
	%\begin{abstract}
	%\noindent \blindtext % Dummy abstract text - replace \blindtext with your abstract text
	%\end{abstract}
}

%----------------------------------------------------------------------------------------

\begin{document}
	
	% Print the title
	\maketitle
	
	%----------------------------------------------------------------------------------------
	%	ARTICLE CONTENTS
	%----------------------------------------------------------------------------------------
	\section{Resumo}
	Esse material ensina atalhos globais e específicos de aplicações, especialmente no linux mint, porém, muitos comandos específicos de sistemas são utilizados da mesma forma em outras aplicações similares mesmo em outras distribuições do linux ou mesmo em outros sistemas operacionais.
	
	\section{Introdução}
	O uso do shell para realizar atividades repetitivas (ou com algum padrão definido) se provou ser bem mais eficiente que o sistema de apontar e clicar, usado como método principal de entrada de comandos na maioria das interfaces mais usadas atualmente. Apesar do mouse ser muito simples e intuitivo de usar, é comum que usuários mais experientes busquem modos mais rápidos de realizar as mesmas ações. Felizmente, é possível identificar padrões nos principais atalhos do teclado em diferentes interfaces, muitas interfaces, usam o mesmo atalho para ações similares, independente do sistema, e algumas vezes é possível inferir um comando pela inicial da ação a ser realizada. Outras aplicações permitem a customização de atalhos, para que o usuário possa inserir atalhos para os comandos mais usados.
	
	\pagebreak
	
	\section{Atalhos Globais}
	Os atalhos da tabela \ref{table:1} são os mais comuns para qualquer tipo de inteface:
	\begin{table}
		\centering
		\begin{tabular}{|p{2.2cm}|p{3cm}|p{8cm}|}
			\hline
			\bfseries Combinação de teclas & \bfseries Função & \bfseries interfaces \\ \hline
			ctrl tab & Muda o foco para o próximo elemento & Navegadores, navegador de arquivos e diretórios, interfaces com várias abas ou vários campos de seleção. \\ \hline
			ctrl shift tab & Muda o foco para o elemento anterior & Navegadores, navegador de arquivos e diretórios, interfaces com várias abas ou vários campos de seleção \\ \hline
		\end{tabular}
		\caption{A tabela mostra a combinação das teclas, suas respectivas funções e interfaces onde podem ser utilizados}
		\label{table:1}
	\end{table}
	
	\section{conclusão}
	

\bibliographystyle{apalike}
\bibliography{artbib}

%----------------------------------------------------------------------------------------
%   REFERENCE LIST
%----------------------------------------------------------------------------------------
\end{document}